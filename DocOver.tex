Dit document behandeld de opslag van data op de verschillende opslagsystemen voor het middelbaar beroepsonderwijs in Nederland.

\section*{Versienummering}
Het versienummer van elk document bestaat uit drie nummers gescheiden door een punt. Het eerste nummer is het major-versie nummer, het tweede nummer het minor-versienummer en de laatste is de nummering voor bug-fixes.\par
Om met de laatste te beginnen als er in het document slechts verbeteringen zijn aangebracht die te maken hebben met type-fouten, websites die niet meer beschikbaar zijn, of kleine foutjes in de opdrachten dan zal dit nummer opgehoogd worden. Als docent of student hoef je je boek niet te vervangen. Het is wel handig om de wijzigingen bij te houden.\par
Als er flink is geschreven aan het document dan zal het minor-nummer opgehoogd worden, dit betekent dat er bijvoorbeeld plaatjes zijn vervangen of geplaatst/weggehaald, maar ook dat paragrafen zijn herschreven, verwijderd of toegevoegd, zonder dat de daadwerkelijk context is veranderd. Een nieuw cohort wordt aangeraden om met deze nieuwe versie te beginnen, bestaande cohorten kunnen doorwerken met het boek dat ze al hebben.\par
Als het major-nummer wijzigt dan betekent dat dat de inhoud van het boek substantieel is gewijzigd om bijvoorbeeld te voldoen aan een nieuw kwalificatiedossier voor het onderwijs. Een nieuw major-nummer betekent bijna altijd voor het onderwijs dat men in het nieuwe schooljaar met deze nieuwe versie aan de slag zou moeten gaan. Voorgaande versies van het document zullen nog tot het einde een schooljaar onderhouden worden, maar daarna niet meer.

\section*{Document ontwikkeling}
Het doel is door middel van open documentatie een document aan te bieden aan zowel studenten als docenten, zonder dat hier hoge kosten aan verbonden zijn en met de gedachte dat we samen meer weten dan alleen. Door samen te werken kunnen we meer bereiken.\par
Bijdragen aan dit document worden dan ook met alle liefde ontvangen. Let u er wel op dat materiaal dat u bijdraagt onder de CC BY-NC-SA licentie vrijgegeven mag worden, dus alleen origineel materiaal of materiaal dat al vrijgegeven is onder deze licentie.\par
De eerste versie is geschreven voor het ROC Horizon College.
