\documentclass[a4paper,12pt,twoside,openright,titlepage]{book}

%Additional packages
\usepackage[ascii]{inputenc}
\usepackage[T1]{fontenc}
\usepackage[dutch,english]{babel}
\usepackage{imakeidx}
\usepackage{syntonly}
\usepackage[official]{eurosym}
%\usepackage[graphicx]
\usepackage{graphicx}
\graphicspath{ {./images/} }
\usepackage{float}
\usepackage{hyperref}
\hypersetup{colorlinks=true, linkcolor=blue, citecolor=blue, filecolor=blue, urlcolor=blue, pdftitle=, pdfauthor=, pdfsubject=, pdfkeywords=}
\usepackage{tabularx}
\usepackage{scrextend}
\addtokomafont{labelinglabel}{\sffamily}
\usepackage{listings}
\usepackage{adjustbox}
\usepackage{color}

% Turn on indexing
\makeindex

% Define colors
\definecolor{ashgrey}{rgb}{0.7, 0.75, 0.71}

% Listing style
\lstset{
  backgroundcolor=\color{ashgrey},   % choose the background color; you must add \usepackage{color} or \usepackage{xcolor}; should come as last argument
  basicstyle=\footnotesize,        % the size of the fonts that are used for the code
  breakatwhitespace=false,         % sets if automatic breaks should only happen at whitespace
  breaklines=true,                 % sets automatic line breaking
  extendedchars=true,              % lets you use non-ASCII characters; for 8-bits encodings only, does not work with UTF-8
  frame=single,	                   % adds a frame around the code
  keepspaces=true,                 % keeps spaces in text, useful for keeping indentation of code (possibly needs columns=flexible)
  rulecolor=\color{black},         % if not set, the frame-color may be changed on line-breaks within not-black text (e.g. comments (green here))
  showspaces=false,                % show spaces everywhere adding particular underscores; it overrides 'showstringspaces'
}

% Uncomment for production
% \syntaxonly

% Style
\pagestyle{headings}

%%%%%%%%%%%%%%%%%%
% Begin document %
%%%%%%%%%%%%%%%%%%

% Define document
\author{D. Leeuw}
\title{Virtualisatie en The Cloud}
\date{\today\\v.0.1.0}

\begin{document}
\selectlanguage{dutch}

\maketitle

\copyright\ 2021 Dennis Leeuw\\

\begin{figure}
\includegraphics[width=0.3\textwidth]{CC-BY-SA-NC.png}
\end{figure}

\bigskip

Dit werk is uitgegeven onder de Creative Commons BY-NC-SA Licentie en laat anderen toe het werk te kopi\"eren, distribueren, vertonen, op te voeren, en om afgeleid materiaal te maken, zolang de auteurs en uitgever worden vermeld als maker van het werk, het werk niet commercieel gebruikt wordt en afgeleide werken onder identieke voorwaarden worden verspreid.


%%%%%%%%%%%%%%%%%%%
%%% Introductie %%%
%%%%%%%%%%%%%%%%%%%

\frontmatter
\chapter{Over dit Document}
Dit document behandeld de opslag van data op de verschillende opslagsystemen voor het middelbaar beroepsonderwijs in Nederland.

\section*{Versienummering}
Het versienummer van elk document bestaat uit drie nummers gescheiden door een punt. Het eerste nummer is het major-versie nummer, het tweede nummer het minor-versienummer en de laatste is de nummering voor bug-fixes.\par
Om met de laatste te beginnen als er in het document slechts verbeteringen zijn aangebracht die te maken hebben met type-fouten, websites die niet meer beschikbaar zijn, of kleine foutjes in de opdrachten dan zal dit nummer opgehoogd worden. Als docent of student hoef je je boek niet te vervangen. Het is wel handig om de wijzigingen bij te houden.\par
Als er flink is geschreven aan het document dan zal het minor-nummer opgehoogd worden, dit betekent dat er bijvoorbeeld plaatjes zijn vervangen of geplaatst/weggehaald, maar ook dat paragrafen zijn herschreven, verwijderd of toegevoegd, zonder dat de daadwerkelijk context is veranderd. Een nieuw cohort wordt aangeraden om met deze nieuwe versie te beginnen, bestaande cohorten kunnen doorwerken met het boek dat ze al hebben.\par
Als het major-nummer wijzigt dan betekent dat dat de inhoud van het boek substantieel is gewijzigd om bijvoorbeeld te voldoen aan een nieuw kwalificatiedossier voor het onderwijs. Een nieuw major-nummer betekent bijna altijd voor het onderwijs dat men in het nieuwe schooljaar met deze nieuwe versie aan de slag zou moeten gaan. Voorgaande versies van het document zullen nog tot het einde een schooljaar onderhouden worden, maar daarna niet meer.

\section*{Document ontwikkeling}
Het doel is door middel van open documentatie een document aan te bieden aan zowel studenten als docenten, zonder dat hier hoge kosten aan verbonden zijn en met de gedachte dat we samen meer weten dan alleen. Door samen te werken kunnen we meer bereiken.\par
Bijdragen aan dit document worden dan ook met alle liefde ontvangen. Let u er wel op dat materiaal dat u bijdraagt onder de CC BY-NC-SA licentie vrijgegeven mag worden, dus alleen origineel materiaal of materiaal dat al vrijgegeven is onder deze licentie.\par
De eerste versie is geschreven voor het ROC Horizon College.

\begin{flushleft}
\begin{table}[h!]
\centering
\begin{tabularx}{\textwidth}{ |c|c|c|X| }
\hline
	Versienummer &
	Auteurs &
	Verspreiding &
	Wijzigingen\\
\hline
	0.1.0 &
	Dennis Leeuw &
	Initieel document\\
\hline
\hline
\end{tabularx}
\caption{Document wijzigingen}
\label{table:1}
\end{table}
\end{flushleft}



%%%%%%%%%%%%%%%%%
%%% De inhoud %%%
%%%%%%%%%%%%%%%%%
\tableofcontents

\mainmatter
\chapter{Simulatie, Emulatie en Virtualisatie}\index{Simulatie}\index{Emulatie}\index{Virtualisatie}
Simulatie, emulatie, imitatie en virtualisatie zijn termen die min of meer hetzelfde betekenen in het Nederlands, maar in de IT een verschillende betekenis kunnen hebben. Vergelijk ook met de Engelse woorden simulation, emulation, imitation en virtualization.

Volgens Van Dale is simulatie het verrichten van schijnbare handelingen of ook nabootsen. Nabootsen komen we ook tegen bij imitatie, terwijl bij emulatie iets staat dat niets te maken heeft met wat we in de IT bedoelen. Van Dale geeft er wedijver en naijver, kortom het beter willen doen. Bij virtueel komt komt in Van Dale iets voor waar we echt wat mee kunnen: Slechts schijnbaar bestaand. Zo zie je dat deze termen snel verwarring kunnen opleveren.

Simulation in de IT is wat er bijvoorbeeld in games gebeurd. Een flight simulator doet een vliegtuig na en geeft je het gevoel echt in een vliegtuig te zitten. Veel andere spelletjes zijn ook simulatoren.

Virtualization is het ophakken van bestaande resources in kleine stukjes. Een virtual machine krijgt zo een klein stukje processor gebruik, een klein stukje geheugen, een stukje van een harddisk. 

Emulation doet hetzelfde als virtualization, maar heeft daarnaast de mogelijkheid om een vertaalslag te maken naar ander type hardware of software. Een emulator geeft je bijvoorbeeld de mogelijkheid om een PowerPC chip na te doen op een ix86 processor. De VM denkt dat zijn processor een PowerPC chip is, terwijl er in de onderliggend fysieke hardware bijvoorbeeld een Core i7 van Intel zit.

Dit document behandeld alleen virtualisatie.

\section{Wat is virtualisatie?}
Virtualisatie is dat iets lijkt te zijn wat het niet is. Op Windows-systemen kan bijvoorbeeld een drive gemapd zijn naar de letter G. Dit lijkt een lokale disk, maar in werkelijkheid is het een netwerk-share.

Virtualisatie in de ICT is een heel breed onderwerp dat gaat van opslag tot servers en van desktops tot applicatie containers. Het doel van  dit document is meer helderheid te verschaffen in de terminologie die gebruikt wordt in dit vakgebied en door middel van opdrachten de lezer meer vertrouwd te maken met de technologie.



\chapter{The Cloud}\index{Cloud}\index{The Cloud}
Tegenwoordig spreken we van cloud oplossingen en van data opslaan in de cloud, maar wat is dat eigenlijk die cloud? De naam cloud komt vermoedelijk van het feit dat in de automatisering als we een netwerk in generieke zin willen weergeven we meestal een wolkje tekenen. We diensten die in het netwerk draaien gaan aanduiden als cloud diensten. Het Internet is het belangrijkste netwerk dat we vaak als cloud beschrijven. Cloud diensten draaien dan ook bij een provider ergens op het Internet. Voor de gebruiker is het niet van belang waar die dienst is.

Bij providers kunnen verschillende soorten diensten worden afgenomen: IaaS, PaaS en SaaS.

\section{IaaS}\index{IaaS}\index{Infrastructure as a Service}
Infrastructure as a Service is een dienst die gevirtualiseerde hardware aanbiedt. Je kan dan bijvoorbeeld kiezen voor 2 CPUs, 2 Gig RAM en 300 GB storage met eventueel nog wat eisen en wensen aan het netwerkverkeer dat naar de machine gaat.

Deze dienst is ook bekend onder de term Virtual Private Server.

Meestal kun je een OS kiezen uit een lijst van beschikbare OSen welke dan op je VPS ge\"installeerd wordt. De rest van de configuratie en installatie van de software moet je zelf doen. Je hebt de volledige vrijheid bij de inrichting.

\section{PaaS}\index{PaaS}\index{Platform as a Service}
Platform as a Service is een dienst die je naast een virtuele machine en een besturingssysteem ook de middleware aanbiedt. Je kan bijvoorbeeld een PaaS dienst bestellen met LAMP. Linux, Apache, MySQL en PHP zullen dan al voor je ge\"installeerd zijn en daar hoef je dus niets meer aan te doen. Je bent alleen nog verantwoordelijk voor het installeren van de applicatie (web-applicatie).

PaaS wordt vaak aangeboden onder de naam webhosting.

\section{SaaS}\index{SaaS}\index{Software as a Service}
Software as a Service biedt software aan via het Internet. Je hoeft niets meer te installeren en heb meestal alleen toegang tot de web-interface van de applicatie. Voorbeelden zijn bijvoorbeeld de Google Applicatie Suite en Office 365.


\chapter{Machine Virtualisatie (IaaS)}\index{VM}\index{Virtual Machine}\index{IaaS}
Machine virtualisatie zorgt ervoor dat er effici\"enter gebruik wordt gemaakt van computing resources. Moderne computers zijn zo krachtig dat ze met \'e\'en besturingssystemen en een paar applicaties of services vaak meer niets staan te doen dan dat ze werken. Door op deze hardware meerdere besturingssystemen te draaien kunnen de systemen effici\"enter ingezet worden. De grote datacenters hebben dan minder hardware nodig wat bezuinigt op stroomgebruik, warmte ontwikkeling beperkt en kastruimte bespaart.

Om meerdere besturingssystemen op \'e\'en machine te draaien is er iets nodig dat de bestaande hardware ophakt in stukjes zodat deze gezien wordt door de verschillende OSen als 'eigen' hardware. De techniek die daarvoor gebruikt wordt heet een hypervisor.

\section{Hypervisor}\index{Hypervisor}
Een hypervisor is een stuk software die hardware virtualisatie mogelijk maakt. Er bestaan twee type hypervisors; type 1 en type 2.

Hypervisor type 1 is een hypervisor die op bare metal draait. De hypervisor levert zijn eigen besturingssysteem en maakt rechtstreeks gebruik van de beschikbare hardware om deze aan te bieden aan de virtuele machines.

De andere variant is een Hypervisor type 2, deze draait op een bestaand besturingssysteem. Je hypervisor is dus eigenlijk een applicatie op een besturingssysteem en maakt gebruik van het onderliggende besturingssysteem om hardware aan te bieden aan de virtuele machines. Deze laatste techniek is simpeler omdat een type 2 hypervisor gebruik kan maken van de drivers van het OS waarop het draait.

Een hypervisor verdeelt de bestaande hardware op in stukjes en biedt deze aan als een nieuwe machine. De hypervisor doet dus alsof hij allemaal zelfstandige computers aanbiedt, vandaar dat we dat virtuele machines noemen (virtual machines), wat me meestal afkorten als VM. De machine (en het OS) waarop de hypervisor draait wordt de Host genoemd, de virtuele machine die op de hypervisor draait wordt de Guest genoemd.

Het is natuurlijk een hele kunst om alle hardware te ondersteunen en de calls van het Guest OS correct door te zetten naar de hardware. Een andere optie is dat het Guest OS niet een echte netwerkkaart driver gebruikt, maar een pseudo-driver die zich bijvoorbeeld voordoet als een netwerkkaart. We hebben het dan over een vorm van emulatie, binnen de virtualisatie wereld noemen we dat paravirtualisatie. Voor paravirtualisatie is het nodig om het OS aan te passen aan de hypervisor. Dit is bijvoorbeeld noodzakelijk als de hardware onder de hypervisor geen virtualisatie ondersteund.

Bijna alle moderne computers en laptops ondersteunen hardware virtualistie. In de BIOS kan de ondersteuning voor virtualisatie aan of uit gezet worden. De twee grote processor makers van onze wereld, Intel en AMD, hebben de techniek van virtualisatie ondersteuning beide een iets ander naam gegeven. De ondersteuning voor Intel processoren heet VT-x en voor AMD heet het AMD-V.

\subsection{Hypervisor Opdracht}
Herstart je laptop en ga naar de BIOS settings van je computer. Zoek in de BIOS waar de settings staan voor virtualisatie en als deze uit staan zet ze dan aan.

\section{Hypervisors}
\subsection{VirtualBox}\index{VirtualBox}
\subsection{VMWare}\index{VMWare}
\subsection{Citrix}\index{Citrix}
\subsection{Xen}\index{Xen}
\subsection{KVM}\index{KVM}
\subsection{Hyper-V}\index{Hyper-V}
\subsection{Parallels}\index{Parallels}
\section{Server Virtualisatie}\index{Server virtualisatie}
\subsection{High Availibility}\index{High Availibility}
\section{Desktop Virtualisatie}\index{Desktop virtualisatie}
\section{The Cloud: Virtual Private Servers}\index{VPS}\index{Virtual Private Server}

\chapter{Webhosting (PaaS)}\index{PaaS}\index{Webhosting}
\section{LAMP}\index{LAMP}

\chapter{Applicatie Virtualistie (SaaS)}\index{SaaS}\index{Applicatie virtualisatie}
Steeds meer diensten draaien in de cloud, om deze diensten aan te bieden draaien er op servers web-servers, database-servers en waarschijnlijk nog allerlei andere services die zorgen dat de totale infrastructuur draaien blijft.

\section{Hosted applications}\index{Hosted applications}
\section{Containerisatie}\index{Containers}
\section{Containerisatie software}
\subsection{LXC}\index{LXC}
\subsection{Docker}\index{Docker}
\subsection{Kubernetes}\index{Kubernetes}
\section{Cloud Applicaties}
\subsection{Google Apps}
\subsection{Microsoft Office 365}

%%%%%%%%%%%%%%%%%%%%%
%%% Index and End %%%
%%%%%%%%%%%%%%%%%%%%%
%\backmatter
\printindex
\end{document}

%%% Last line %%%
