Simulatie, emulatie, imitatie en virtualisatie zijn termen die min of meer hetzelfde betekenen in het Nederlands, maar in de IT een verschillende betekenis kunnen hebben. Vergelijk ook met de Engelse woorden simulation, emulation, imitation en virtualization.

Volgens Van Dale is simulatie het verrichten van schijnbare handelingen of ook nabootsen. Nabootsen komen we ook tegen bij imitatie, terwijl bij emulatie iets staat dat niets te maken heeft met wat we in de IT bedoelen. Van Dale geeft er wedijver en naijver, kortom het beter willen doen. Bij virtueel komt komt in Van Dale iets voor waar we echt wat mee kunnen: Slechts schijnbaar bestaand. Zo zie je dat deze termen snel verwarring kunnen opleveren.

Simulation in de IT is wat er bijvoorbeeld in games gebeurd. Een flight simulator doet een vliegtuig na en geeft je het gevoel echt in een vliegtuig te zitten. Veel andere spelletjes zijn ook simulatoren.

Virtualization is het ophakken van bestaande resources in kleine stukjes. Een virtual machine krijgt zo een klein stukje processor gebruik, een klein stukje geheugen, een stukje van een harddisk. 

Emulation doet hetzelfde als virtualization, maar heeft daarnaast de mogelijkheid om een vertaalslag te maken naar ander type hardware of software. Een emulator geeft je bijvoorbeeld de mogelijkheid om een PowerPC chip na te doen op een ix86 processor. De VM denkt dat zijn processor een PowerPC chip is, terwijl er in de onderliggend fysieke hardware bijvoorbeeld een Core i7 van Intel zit.

Dit document behandeld alleen virtualisatie.
