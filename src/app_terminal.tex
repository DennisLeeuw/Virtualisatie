In de jaren 1960 en 1970 bestonden er nog geen Personal Computers, laptops en mobiele telefoons. Er waren centrale computers, mainframes en mini's, die met terminals verbonden waren. Een terminal was niets anders dan een beeldscherm en een toetsenbord op afstand. Characters werden van de centrale computer verstuurd naar de terminal en toetsenbord aanslagen werden vestuurd van de terminal naar de computer. Grafische schermen bestonden ook nog niet. De weinige gebruikers die met een computer mochten werken werkten dus op de centrale computer. Alle applicaties waren centraal, alle data was centraal.

In de jaren 1980 werd de Personal Computer uitgevonden en daarmee verplaatsten de data en de applicaties zich naar het apparaat van de gebruiker.
