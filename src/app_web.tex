Een moderne vorm van gehoste applicaties is de web applicatie. Een programma geschreven in bijvoorbeeld PHP draait op de webserver. Op de client draait een browser en tussen de client en de browser stuurt HTTP toetsenbord-informatie  en muis-clicks naar de server en HTML en CSS (schermopmaak) naar de gebruiker. Het is een complete client-server setup waar we allemaal dagelijks gebruik van maken.

De meest gebruikte web-applicatie is waarschijnlijk de webshop. Gebruikers vragen een pagina op, lopen door een menu, klikken op een product en rekenen af bij de kassa. Het is inmiddels zo normaal geworden dat we er meestal niet meer aan denken wat er werkelijk gebeurt. Op het moment dat je op een product of een menu item klikt vertaalt je browser je muis-positie in een URL en vraagt deze op bij de server, of als je op bestellen klikt vertaalt je browser je muis-positie naar de URL van de button die je aanklikte. Vervolgens krijg je een scherm waar je je gegevens moet invullen. Toetsenbord aanslagen worden omgezet naar leesbare tekst in een invoer veldje etc. En bij elke klik op een button komt er een nieuw scherm naar je toe. Verpakt in HTML elementen wordt er data naar je gestuurd waar je browser een grafisch plaatje van maakt.

Het grote voordeel van web-applicaties is dat de gebruiker alleen nog afhankelijk is van de browser. Het maakt niet meer uit op welk operating system de applicatie draait en voor de gebruiker maakt het niet meer uit welk besturingssysteem deze ge\"installeerd heeft. Er kunnen natuurlijk nog wel afhankelijkheden zijn van een bepaalde browser.

Je applicatie draait dus in The Cloud, of wel op het Internet, je data vaak ook. Je data staat niet meer lokaal, maar bij een provider. Op je Google Drive of je Microsoft OneDrive, maar ook bij de kleinere applicatie providers staan je data niet meer op je systemen, in je eigen organisatie. Het heeft zijn voordelen om niet meer zelf verantwoordelijk te zijn voor de beveiliging van je data, maar hoe goed vertrouw je de bedrijven? Kunnen zij je vertrouwelijke bedrijfsdata lezen?
