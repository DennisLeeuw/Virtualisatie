Een moderne vorm van gehoste applicaties is de web applicatie. Een programma geschreven in bijvoorbeeld PHP draait op de webserver. Op de client draait een browser en tussen de client en de browser stuurt HTTP toetsenbord-informatie  en muis-clicks naar de server en HTML en CSS (schermopmaak) naar de gebruiker. Het is een complete client-server setup waar we allemaal dagelijks gebruik van maken.

De meest gebruikte web-applicatie is waarschijnlijk de webshop. Gebruikers vragen een pagina op, lopen door een menu, klikken op een product en rekenen af bij de kassa. Het is inmiddels zo normaal geworden dat we er meestal niet meer aan denken wat er werkelijk gebeurd. Op het moment dat je een product om een menu item aanklikt vertaald je browser je muis-positie in een URL en vraagt deze op bij de server, of als je op bestellen klikt vertaald je browser je muis-positie naar de URL van de button die je aanklikte. Vervolgens krijg je een scherm waar je je gegevens moet invullen. Toetsenbord aanslagen worden omgezet naar leesbare tekst in een invoer veldje etc. En bij elke klik op een button komt er een nieuw scherm naar je toe. Verpakt in HTML elementen wordt er data naar je gestuurd waar je browser een grafische plaatje voor je van maakt.

Web-applicaties zoals Microsoft Office 365 of Google Workspace zijn de moderne variant op het hosted applicatie concept.
