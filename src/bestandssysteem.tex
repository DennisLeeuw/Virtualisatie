Een hardeschijf heet een block device omdat data gelezen (en geschreven) kan worden in vaste groottes, genaamd blocks, sectors of clusters. Een block is meestal 512 bytes of een veelvoud daarvan. Je kan data rechtstreeks naar een block device sturen, maar meestal wordt een block device geformatteerd zodat er een bestandssysteem wordt aangemaakt. Een bestandssysteem is in het Engels een filesystem en daarvan bestaan er verschillende. Windows gebruikt tegenwoordig NTFS, maar vroeger het FAT-filesystem. Linux kent het extended filesystem (extfs), maar ook bijvoorbeeld BTRFS en Reiserfs. Apple gebruik het Apple File System (APFS).

Je kan geen Apple APFS disk lezen op een Windows systeem en ook geen Windows NTFS disk op de Mac. Onder Linux is NTFS-3G ontwikkeld en daarmee kan Linux NTFS disks lezen en schrijven. NTFS-3G is ook beschikbaar voor Mac OS X. Er is experimentele ontwikkeling gaande om ook APFS te ondersteunen onder Linux. FAT is op bijna elk systeem lees- en schrijfbaar, daar dit systeem al heel oud is en er voor bijna elk besturingssysteem wel drivers voor zijn.

