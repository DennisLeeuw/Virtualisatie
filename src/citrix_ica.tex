De Independent Computing Architecture bevat software op de server, een netwerk protocol en software op de client. Het ICA systeem verplaatst bitmaps, of gedeelten van bitmaps, van de server naar de applicatie en stuurt muis pointer informatie en toetsenbord-data van de client naar de server. Het protocol kan ook applicatie audio, en andere media, naar de client sturen. Het versturen van bitmaps naar de client is een relatief bandbreedte intensief proces. Citrix Systems is dan ook steeds opzoek naar betere methodes om deze overdracht zo goed mogelijk te laten verlopen. De laaste incarnatie heet HDX/ICA.

ICA maakt gebruik van het zogenaamde screen scrapping. Dat is het uitlezen van de grafische kaart buffers en deze informatie versturen over het netwerk. De kunst is om zo min mogelijk data te versturen zodat gebruikers een optimale performance waarnemen. Het minimaliseren kan gebeuren daar alleen de scherm-wijzigingen op te sturen of door de bitmap eerst te comprimeren voordat deze verstuurd wordt. Combinaties van technieken leveren de beste performance op.
