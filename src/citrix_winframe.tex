In 1995 bracht Citrix Systems\index{Citrix Systems} een product met de name WinFrame\index{WinFrame} op de markt. WinFrame was Windows NT 3.51 met een aanpassing om centrale applicaties over het netwerk te draaien. Meerdere gebruikers konden tegelijkertijd ingelogd zijn op het systeem en daar applicaties opstarten. De scherm informatie werd dan naar het werkstation gestuurd. De applicaties zijn dus niet op het werkstation ge\"installeerd, maar alleen op de server. Op het werkstation is een ICA\index{ICA}\index{Indepenent Computing Architecture} client nodig. Het werkstation kan een Windows systeem zijn, maar ook bijvoorbeeld een Mac OS X systeem of Linux laptop. Op deze manier kunnen Windows applicaties gedeeld worden. ICA (Independent Computing Architecture) protocol is het door Citrix System ontworpen protocol voor de communicatie tussen de applicatie en het werkstation.

Om WinFrame te ontwikkelen heeft Citrix een licentie genomen op de Windows broncode voor NT 3.51 en heeft aan deze broncode de aanpassingen gedaan om van Windows NT een multi-user OS te maken. De verbinding tussen WinFrame en de desktop client wordt gemaakt via het ICA protocol. WinFrame heeft in de loop van de tijd wat naamsveranderingen ondergaan. Het heeft onderandere MetaFrame\index{MetaFrame}, Presentation Server\index{Presentation Server} en XenApp\index{XenApp} geheten. Het is tegenwoordig op de markt als Citrix Virtual Apps\index{Citrix Virtual Apps}.

De client applicatie heette vroeger Citrix Receiver\index{Citrix Receiver} en heet nu Citrix Workspace App\index{Workspace App}, voor Linux wordt er vaak gesproken over de ICAClient.

Microsoft heeft de code van Citrix WinFrame gelicenseerd van Citrix en heeft daarmee Microsoft Terminal Server\index{Terminal Server}\index{Microsoft Terminal Server} op de markt gebracht in Windows NT 4.0. Vanaf Windows 2008 R2 (2009) heet het product Remote Desktop Services\index{RDS}\index{Remote Desktop Services}.
