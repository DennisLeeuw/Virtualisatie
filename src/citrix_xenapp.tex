Citrix XenApp geeft je de mogelijkheid om een enkele applicatie (Hosted Application) over het netwerk te draaien of een volledige desktop (Hosted Shared Desktop). Muisklikken en toetsenbord aanslagen gaan van de client naar de server en scherm verversingen gaan van de server naar de client. Aan de client kant is de Citrix client software nodig, aan de server kant de XenApp server. De communicatie verloopt via het ICA protocol. Deze manier van werken is compleet onafhankelijk van het client OS. Het enige dat van belang is is de Citrix client.

Een andere optie van XenApp is Application Streaming. Als de applicatie gestreamd wordt naar de client dan draait deze op de client en gebruikt resources van de client. Je kan de applicatie ook streamen naar een XenApp server, waardoor deze weer beschikbaar gemaakt kan worden via ICA. Streamed applicaties worden gecached op de client en hoeven dus maar 1x de gedownload te worden. Als de centraal opgeslagen applicatie geupdate wordt wordt de applicatie op de client opnieuw gedownload en gecached. Streamed applicaties draaien in een sandbox en kunnen dus de bestaande applicatie op de client niet in de weg zitten. Ze draaien wel op het OS van de client, dus zijn daarvan wel afhankelijk.

De XenApp server is afhankelijk van de Windows Remote Desktop Services Session Host Role en deze moet dus geinstalleerd zijn.

De Hosted Applications kunnen op client op verschillende manieren gebruikt worden. Ze kunnen gestart worden vanuit een webinterface (via de Citrix Online Plug-in), via een ICA client op het Client OS of met een ICA client op een zero-client. Voor het streamen van een applicatie is de Citrix Offline Plug-in nodig.
