Een uitbreiding op het VPS concept is de toevoeging van LAMP\index{LAMP}. Naast de virtuele hardware krijg je dan ook de zogenaamde middle ware, duse een omgeving om je software in te draaien vaak bestaat die uit de LAMP-stack ofwel Linux, Apache, MySQL en PHP. Je hoeft hieraan alleen de gewenste applicatie toe te voegen en de data in de database te laden om operationeel te zijn. De VPS en LAMP worden beheerd en onderhouden door je provider.

Op de LAMP-stack zijn natuurlijk vele variaties mogelijk zoals WAMP, waarbij Linux vervangen is door Windows, of MAMP als Mac OS X gebruikt wordt als besturingssysteem. Zo kan de webserver vervangen worden door IIS, NGINX of Cherokee. De database kan naast MySQL MariaDB, PostgreSQL of SQLite zijn en PHP kan ook Perl of Python zijn. De generieke benaming blijft de LAMP-stack omdat nog steeds het meeste voorkomt.

We noemen deze oplossing PaaS, of wel Platform as as Service. Andere PaaS oplossingen zijn met een Java runtime of een .NET runtime. PaaS is een ontwikkel en implementatie omgeving in de Cloud. Vaak wordt in een PaaS oplossing ook de security van het systeem geregeld door de provider. Het is bij deze oplossing wel cruciaal dat de applicatie aansluit bij de versienummers van de runtime die gedraaid wordt en dat updates van die runtime aan blijven sluiten bij de draaiende applicatie.
