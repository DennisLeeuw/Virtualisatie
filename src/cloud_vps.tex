Een VPS, Virtual Private Server, is niets anders dan een virtuele server die als dienst aangeboden wordt door een provider (hosting bedrijf). De VPS draait op een server van een aanbieder ergens in zijn of haar datacenter. Een bekende uitspraak in de cloud wereld is daarom niet voor niets "The cloud that is just somebody else's computer".

Het grote voordeel van het gebruik van een VPS is dat al het beheer van het datacentrum, de computers en de netwerken is uitbesteed (ge-outsourced). Hier kan enorm bespaard worden op salarissen, gebouw ruimte en de steeds terug kerende kosten van het aanschaffen van nieuwe systemen. Providers kunnen door hun schaalgrootte goedkoper inkopen en de kosten kunnen ze daardoor relatief laag houden.

Het nadeel is dat de computers niet meer in de eigen datacentra staan en dat de afhankelijkheid van een goed en betrouwbaar Internet toeneemt. Bij een calamiteit met de Internet verbinding kan een heel bedrijf stil komen te liggen.
