Applicatie containers zijn de nieuwste vorm van virtualisatie. Een applicatie met al zijn afhankelijkheden wordt ingepakt in een pakket en als standalone unit gedeployed. Het idee om applicaties te isoleren van hun omgeving (sandboxing) en zo een veiliger omgeving te cre\"eeren is niet nieuw. Unix systemen kennen die mogelijkheid al vele jaren. Zeker sinds 1979 bestaat daar al de chroot-omgeving\index{chroot}. De chroot-omgeving zorgt ervoor dat een proces niet bij bestanden en processen kan die buiten zijn omgeving liggen. Mocht er op een proces in een chroot-omgeving gebroken worden dat ligt niet meteen het hele systeem open. Met de opkomst van web-applicaties is deze vorm van security steeds belangrijker geworden.

Vanaf 2000 kwam de ontwikkeling van applicatie isolatie in een stroomversnelling. Op FreeBSD ontstond Jails\index{Jails} in 2000 en op Linux ontstond VServer\index{VServer} in 2001. Beide oplossingen deden aan operating system virtualisatie. Het besturingssysteem wordt opgehakt in delen en de applicatie kan gebruik maken van aangeboden resources en all\'e\'en van die aangeboden resources. In 2004 kwam SUN met Solaris Containers\index{Solaris Containers}. Google kwam in 2006 met Process Containers\index{Proces Containers} voor Linux wat uiteindelijk als cgroups\index{cgroups} onderdeel werd van de 2.6.24 kernel van Linux. In 2008 kwam LXC\index{LXC} op de markt, de tot dan toe meest complete container manager. Het maakt gebruik van cgroups en Linux namespaces\index{namespaces}. Vanaf 2013 als Docker\index{Docker} op de markt verschijnt wordt containerisatie echt een hype. Het voordeel van Docker is dat het een totaal oplossing vormt en nog makkelijk in gebruik is ook. De laatste grote stap is de ontwikkeling van Kubernetes\index{Kubernetes}, waar vele grote bedrijven achter zijn gaan staan zoals Amazone, Google, Microsoft, Oracle, VMWare en Red Hat.

In 2015 werd de Cloud Native Computing Foundation\index{Cloud Native Computing Foundation} (CNCF\index{CNCF}) opgericht vanuit de Linux Foundation. Het doel was om ervoor te zorgen dat de neuzen bij de ontwikkeling \'e\'en kant op bleven staan zodat er gezamenlijk gewerkt werd aan \'e\'en doel. Met de oprichting kwam ook Kubernetes 1.0 dat door Google gedoneerd was aan de Linux Foundation. Kubernetes is een open source cluster manager, net zoals Docker dat is. In 2017 doneert Docker containerd aan de CNCF en containerd groeit uit tot \'en\'en van de belangrijkste runtime engines voor containers, een andere is cri-o een container runtime voor kubernetes.

Meer informatie over de geschiedenis van containers is te lezen op \href{https://blog.aquasec.com/a-brief-history-of-containers-from-1970s-chroot-to-docker-2016}{de blog van Aquasec}.
