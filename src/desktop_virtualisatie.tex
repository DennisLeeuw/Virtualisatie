Wat we met servers kunnen, kunnen we natuurlijk ook met desktops. We kunnen een virtuele machine aanmaken en daarop een desktop OS installeren en dan met een applicatie deze desktop overnemen. Hierdoor worden over het netwerk alleen de toetsenbord en muis gegevens opgestuurd naar de virtuele machine en terug komt de beeldschem-informatie. De client kan dus met veel minder hardware toe.

Voor de beheerder zit de winst erin dat updates alleen nog op de server hoeven te worden uitgevoerd, waarbij hij het zelfs met \'e\'en image af zou kunnen, en dat bij een probleem de gebruiker naar een andere VM gestuurd kan worden terwijl de beheerder de problematische VM opknapt. Ook wordt er op deze manier weer effici\"enter met resources omgegaan. Er hoeft niet meer in elke machine een zware processor en geheugen te zitten, niet gebruikte resources kunnen verdeeld worden over de beschikbare VMs.

Een ander voordeel is dat de desktop op deze manier ook remote aangeboden kan worden aan gebruikers wat tele-werken of thuiswerken makkelijker maakt.
