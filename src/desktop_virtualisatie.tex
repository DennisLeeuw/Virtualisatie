Wat we met servers kunnen, kunnen we natuurlijk ook met desktops. We kunnen een virtuele machine aanmaken en daarop een desktop OS installeren en dan met een applicatie deze desktop overnemen. Over het netwerk worden dan de toetsenbord invoer en muis gegevens opgestuurd naar de virtuele machine en naar de applicatie wordt de beeldschem-informatie gestuurd. De client kan dus met veel minder hardware toe. Ook aan de beheer kant kan er winst geboekt worden. De updates hoeven allen nog op de server te worden uitgevoerd, bij een probleem met een VM kan de gebruiker naar een andere VM gestuurd worden terwijl de beheerder de problematische VM opknapt. Ook wordt er op deze manier weer effici\"enter met resources omgegaan. Er hoeft niet meer in elke machine een zware processor en geheugen te zitten, niet gebruikte resources kunnen verdeeld worden over de beschikbare VMs.

Een ander voordeel is dat de desktop op deze manier ook remote aangeboden kan worden aan gebruikers wat tele-werken of thuiswerken makkelijker maakt.
