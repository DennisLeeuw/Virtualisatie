Het aanbieden van gevirtualiseerde desktops kent zijn nadelen. Voor elke gelijktijdige gebruiker in een organisatie moet er een virtual machine aanwezig zijn. Dit kan oplopen tot honderden of duizenden virtuele machines die via het netwerk ontsloten moeten worden. Dit vraagt veel van de virtualisatie omgeving, maar ook van het netwerk en de opslagsystemen. Naast de gebruikersdata moeten ook alle virtual machine images opgeslagen worden. Het loont dus om deze images zoveel mogelijk uit te kleden, tot het absolute minimum.

De initi\"ele kosten zijn hoog. Er zijn zware servers nodig om de desktops virtueel te kunnen draaien, er zijn extra licenties nodig voor de software en er blijven nog steeds eindgebruikers devices nodig, zoals thin clients of laptops, om de virtuele desktops te kunnen benaderen.

Een ander probleem dat vaak ontstaat is het gebruik van de lokale functionaleiten van het access device. Denk dan bijvoorbeeld aan het gebruik van USB-sticks of het kunnen draaien van muziek. De USB-stick moet gekoppeld kunnen worden aan de remote desktop en het geluid van de remote desktop player moet hoorbaar zijn op het access device.

Een laatste nadeel is het anytime, anyplace, anywhere, probleem. Zonder degelijke Internet verbinding is het gebruik van een remote desktop niet mogelijk, of zo traag dat we er niet op willen werken.
