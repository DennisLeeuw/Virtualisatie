Om het omvallen van systemen op te vangen wordt er gebruik gemaakt van failover oplossingen. We spreken dan van hot- of cold-standby. Een tweede identieke machine kan klaarhangen in de kast, als de primaire server omvalt kan een systeembeheerder het tweede systeem opstarten en zorgen dat de gebruikers weer kunnen werken. Dit levert natuurlijk downtime op, maar als alles goed voorbereid is kan een server in een halfuur weer operationeel zijn. Dit is de cold-standby oplossing, de tweede server draait niet en is dus koud.
Hot-standby betekent dat de failover-server al draaiend op het netwerk aanwezig is. Op het moment dat de primaire server omvalt neemt de tweede server automatisch zijn functies over. De techniek hierachter is meestal het delen van een IP-adres tussen het primaire en het failover systeem. Beide systemen krijgen op hun netwerk interface \'e\'en IP adres toewijzen dat bij de machine hoort en er is \'e\'en IP adres dat ze delen, het zogenaamde virtuele IP adres. Dit virtuele IP adres wordt gebruikt door de gebruikers om een dienst te benaderen. Wie het virtuele IP adres op zijn netwerkkaart heeft staan handelt dus de dienst af. Normaal gesproken zal dat het primaire systeem zijn. Mocht deze echter uitvallen dan neemt het failover systeem het IP adres over op zijn netwerkkaart en krijgt daardoor alle vragen van de gebruikers te verwerken.

\begin{figure}[H]
	\includegraphics[width=\linewidth]{fs_failover.png}
	\caption{File Server Failover}
	\label{FS_failover}
\end{figure}

Twee dingen zijn hierbij van belang, namelijk dat de data op server 1 gelijk is aan die op server 2. In figuur \ref{FS_failover} synchroniseren server 1 en 2 over een (rode) ethernet cross-cable, op deze manier verstoort de syncronisatie niet de bandbreedte naar de gebruikers. Het andere dat van belang is is dat server 2 moet weten wanneer server 1 niet meer bereikbaar is. Dat laatste wordt bereikt met een heartbeat, zeg maar een soort ping waarbij de servers elkaar in de gaten houden. Als server 2 de heartbeat van server 1 niet meer hoort dan neemt het het virtuele IP adres over. De heartbeat kan over een speciale kabel gaan of over de al bestaande netwerken tussen de twee servers.
