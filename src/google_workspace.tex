Google is ooit begonnen als search engine en daarin is het nog steeds de grootste. Het begon in 1996 en de domeinnaam werd geregistreerd in 1997. Twee studenten van Stanford University, Larry Page en Sergey Brin, hadden een manier bedacht om webpaginas in een volgorde te zetten op basis van het aantal links dat naar een pagina verwees. Op basis van deze populariteitsindex brachten ze enige orde in de chaos die Internet is.

1 april 2004 starte de beta van GMail waarvoor je per uitnodiging toegang kon krijgen tot een webgebaseerde e-mail oplossing met een opslag van 1 gigabyte. Sinds 2007 kan iedereen zich aanmelden. De mail kan sinds 2007 ook gelezen worden zonder gebruik te maken van de web-interface, namelijk door gebruik te maken van IMAP of POP3. De dienst wordt gratis aangeboden waarbij de inkomsten voor Google gegenereerd worden door gepersonaliseerde advertenties.

Google ging door met de ontwikkeling van applicaties met een web-interface en momenteel kent de omgeving die officieel de Google Workspace is gaan heten een hele lijst met applicaties:
\begin{itemize}
    \item Gmail
    \item Meet
    \item Chat
    \item Agenda
    \item Drive
    \item Documenten
    \item Spreadsheets
    \item Presentaties
    \item Formulieren
    \item Sites
    \item Keep
    \item Apps Script
    \item Cloud Search
\end{itemize}
