Met alles en iedereen verbonden aan het Internet is het niet meer te zeggen wie wanneer van een dienst gebruik wil maken. De gebruikers kunnen van overal op de wereld komen en dus moeten diensten 24-uur per dag en 7 dagen in de week beschikbaar zijn. Servers mogen niet meer uitvallen en als ze toch uitvallen mag het de dienstverlening niet verstoren. Deze hoge beschikbaarheid of wel high-availibilty is steeds meer de norm, dan de uitzondering.

In de vorige paragrafen hebben we gezien dat de huidige infrastructuur vraagt om schaalbaarheid (scalability) en continu\"iteit (continuity). Deze worden ingegeven door de business ofwel het gaat om business continuity. Het enkele uren down zijn van het netwerk bij een klein advocaten kantoor kan al schadelijk zijn voor de reputatie of zelfs gevolgen hebben voor cli\"enten als de stukken niet optijd zijn ingeleverd bij de rechtbank. In een ziekenhuis kan het levens kosten. De afhankelijkheid van IT in onze samenleving is heel groot en elke keer zal er een afweging gemaakt moeten worden tussen de kosten en de wens om altijd online te zijn.

Om de uptime zo hoog mogelijk te maken en daarmee de availablity wordt vaak gebruik gemaakt van het dubbel uitvoeren van componenten. Switches worden dubbel uitgevoerd, netwerkkabels worden redundant aangesloten, systemen worden in clusters gezet en naast voeding uit het stroomnet worden er UPSen en aggregaten neergezet. Alles om er maar voor te zorgen dat de systemen blijven werken. Sommige bedrijven gaan zelfs zo ver dat de datacenters dubbel worden uitgevoerd.

Het niet leveren van een bepaalde dienstverlening levert downtime op. Dit kan worden opgevangen door systemen of onderdelen van systemen dubbel uit te voeren. Zo kan een computer uitgerust zijn met een redundante voeding en/of harddisk. Redundant betekent dat bijvoorbeeld de voeding niet operationeel is. Alleen als de primaire voeding ermee stopt neemt de redundante of secundaire voeding het over. Deze techniek staat ook bekend als hot-standby. Alles is aangesloten en zonder enige downtime neemt het redundante apparaat de operationele acties over. De systeembeheerder hoeft achteraf alleen het defecte onderdeel te vervangen, zodat ook een volgende calamiteit opgevangen kan worden. Het vervangen van het defecte onderdeel kan gebeuren terwijl het systeem doordraait. We noemen deze onderdelen dan ook hot-swappable. Alles is erop gebouwd dat het systeem niet uitgezet hoeft te worden, zodat downtime wordt vermeden.

