Een hypervisor is een stuk software die hardware virtualisatie mogelijk maakt. Er bestaan twee type hypervisors; type 1 en type 2.

Hypervisor type 1 is een hypervisor die op bare metal draait. De hypervisor levert zijn eigen besturingssysteem en maakt rechtstreeks gebruik van de beschikbare hardware om deze aan te bieden aan de virtuele machines.

De andere variant is een Hypervisor type 2, deze draait op een bestaand besturingssysteem. De hypervisor is dus eigenlijk een applicatie op een besturingssysteem en maakt gebruik van het onderliggende besturingssysteem om hardware aan te bieden aan de virtuele machines. Deze laatste techniek is simpeler omdat een type 2 hypervisor gebruik kan maken van de drivers van het OS waarop het draait.

Een hypervisor verdeelt de bestaande hardware op in stukjes en biedt deze aan als een nieuwe machine. De hypervisor doet dus alsof hij allemaal zelfstandige computers aanbiedt, vandaar dat we dat virtuele machines noemen (virtual machines), wat me meestal afkorten als VM. De machine (en het OS) waarop de hypervisor draait wordt de Host genoemd, de virtuele machine die op de hypervisor draait wordt de Guest genoemd.

Het is natuurlijk een hele kunst om alle hardware te ondersteunen en de calls van het Guest OS correct door te zetten naar de hardware. Een andere optie is dat het Guest OS niet een echte netwerkkaart driver gebruikt, maar een pseudo-driver die zich bijvoorbeeld voordoet als een netwerkkaart. We hebben het dan over een vorm van emulatie, binnen de virtualisatie wereld noemen we dat paravirtualisatie. Voor paravirtualisatie is het nodig om het OS aan te passen aan de hypervisor. Dit is bijvoorbeeld noodzakelijk als de hardware onder de hypervisor geen virtualisatie ondersteund.

Bijna alle moderne computers en laptops ondersteunen hardware virtualistie. In de BIOS kan de ondersteuning voor virtualisatie aan of uit gezet worden. De twee grote processor makers van onze wereld, Intel en AMD, hebben de techniek van virtualisatie ondersteuning beide een iets ander naam gegeven. De ondersteuning voor Intel processoren heet VT-x en voor AMD heet het AMD-V.
