Als je een operating system installeerd op een hypervisor dan wordt dit ge\"installeerd in een bestand. Een werkelijke harddisk is niet aanwezig in een VM, dus een bestand wordt gebruikt als virtuele harddisk. Deze virtuele harddisk met daarop een ge\"installeerd OS heet een image. Images kunnen op verschillende manieren gemaakt worden.

Als van een harddisk de data bit voor bit gelezen wordt en weggeschreven wordt als bestand dan heet dat bestand een raw-image. Het bestand zal dus net zo groot zijn als de oorspronkelijke grote van de harddisk. Dit neemt veel ruimte in en er kan effici\"enter gebruik gemaakt worden door het bestand bijvoorbeeld te zippen. Naast een image van een harde schijf kan er ook een image gemaakt worden van een CD- of DVD-disk. Deze images worden ISO\index{ISO-image} images genoemd en hebben de \texttt{.iso} extensie.

Verschillende virtualisatie systemen hebben verschillende oplossingen bedacht om minder schijfruimte in te nemen, maar niet de overhead van zippen te hebben. Dit heeft verschillende image formaten opgeleverd.
