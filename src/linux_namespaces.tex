Namespaces kwamen voor het eerst voor in de Linux kernel versie 2.4.19 uit 2002. Met namespaces kan je een resource groep maken waarbij de wereld er volgens alle leden van de groep hetzelfde uit ziet, maar een andere groep kan een heel ander beeld hebben.
Stel dat we een namespace webserver hebben met daarin een Apache webserver een MySQL database. Beide gebruiken dezelfde netwerkkaart en IP-adressen, maar op diezelfde server kan ook een SAMBA-server draaien in een andere namespace met totaal andere IP-adressen. De beide namespaces kunnen op deze manier ge\"isoleerd van elkaar op dezelfde server draaien.
