De installatie van Windows clients op VMs is natuurlijk ook aan licentiebeheer onderhevig. Het feit dat een OS ge\"installeerd wordt op een VM maakt het niet licentie vrij. De licentie is niet dezelfde als de OEM (Original Equipment Manufacturer) licentie, we hebben namelijk geen leverancier van apparatuur. Voor het gebruik van Windows op virtual machines is er de VDA CAL (Virtual Desktop Access Client-Access Licentie). Dit is een licentie per actieve-workload. Het gaat er dus om om te weten hoeveel systemen er simultaan moeten kunnen draaien.

Voor het beheer van de licenties op de virtuele machines is er een licentieserver nodig die als er een virtuele machine wordt opgestart een licentie uit de beschikbare pool ter beschikking stelt. Als er meer VMs worden gestart dan er licenties zijn, zullen de extra machines niet gebruikt kunnen worden tot er weer een licentie vrij gegeven wordt.
