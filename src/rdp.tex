De afkorting RDP\index{RDP} staat voor Remote Desktop Protocol\index{Remote Desktop Protocol} het is het protocol dat tussen de client en de server wordt gebruikt. De server kan ook een ander desktop OS zijn, voor het gemak praten we over de server als we het hebben over de machine die de desktop aanbiedt, de client is dan de machine de een verbinding maakt met de server. Op de client moet software draaien die het Remote Desktop Protocol praat. We noemen de software op de client de Remote Desktop Client\index{Remote Desktop Client} of sinds Windows 10 Remote Desktop Connection\index{Remote Desktop Connection}, de afkorting is voor beide hetzelfde: RDC\index{RDC}. Het geheel van server software, client software en het netwerk protocol heet de Remote Desktop Service, afgekort: RDS.

