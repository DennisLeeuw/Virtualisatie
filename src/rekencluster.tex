Rekenclusters zijn verzamelingen van machines die gebruikt worden om complexe rekentaken uit te voeren. We kunnen rekenclusters onderscheiden in twee smaken, de clusters die voornamelijk rekenen met integers (dus gehele getallen) en de clusters die voornamelijk rekenen met floats (dus getallen met cijfers achter de komma). Voor de eerste rekenclusters volstaan meestal de standaard processoren. De rekenclusters die met floats werken hebben vaak voordeel bij het gebruik van GPU's als rekenunits. De Engelse term voor een rekencluster is High Performance Cluster en de techniek heet dan ook High Performance Computing.

Omdat de wensen per rekenopdracht sterk kunnen verschillen, veel CPU power of just veel geheugen, worden de nodes van een rekencluster vaak gevirtualiseerd aangeboden zodat er snel en effici\"ent geschaald kan worden. De overhead van de hypervisor is minder belangrijk dan de schaalbaarheid van het totale cluster waarbij nodes flexibel toegevoegd of verwijderd kunnen worden aan bepaalde rekentaken.
