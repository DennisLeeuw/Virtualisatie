VDI staat voor Virtual Desktop Infrastructure en is een uitvinding van VMware dat het in 2006 op de markt bracht. VMWare gebruikte zijn eigen technologie, VMWare ESXi, om daarop Windows XP virtual machines te laten draaien. Via de standaard Remote Desktop software van Microsoft waren deze desktops beschikbaar voor gebruikers. Het is een techniek om desktop systemen beschikbaar te stellen, waarbij de desktop computers vervangen kunnen worden door thin clients.

De clients verbinden aan de VM via een remote desktop protocol zoals Microsofts Remote Desktop Protocol (RDP) of Citrix ICA (Independent Computing Architecture) protocol. De protocollen sturen scherm afbeeldingen naar de client en de muis en toetsenbord gegevens naar de VM.

De Virtual Desktop Environment (VDE) draait volledig op de server en ook alle data blijft op de server. Een voordeel van VDI is dan ook dat als een eindgebruikersdevice wordt gestolen, dat alle gegevens op de server staan. Er kan dus bij diefstal geen of minder snel gevoelige informatie op straat komen te liggen. Een ander voordeel is dat omdat de VDE op de server draait de toegang tot deze omgeving niet gelimiteerd hoeft te zijn tot het bedrijfspand, maar de desktop omgevingen ook beschikbaar gesteld kunnen worden aan thuisgebruikers via bijvoorbeeld een VPN (Virtual Private Network) verbinding.

VDI bestaat er in 2 smaken, persistent VDI en non-persistent VDI:
\begin{itemize}
\item Bij persistent VDI gedraagt de desktop zich zoals we ook van een installatie op een eigen machine verwachten. De wijzgigingen aan de achtergrond, de kleuren en andere zaken worden opgeslagen op het systeem. Elke keer als de gebruiker inlogd op een persisten VDI desktop ziet zijn omgeving eruit zoals deze het heeft achtergelaten. Hte feit dat wijzigingen worden opgeslagen betekent wat we het hebben over een personal image of personal workload. Het is natuurlijk niet handig om per gebruiker een persoonsafhankelijke image op te slaan, dat zou veel te veel ruimte kosten. Voor desktopsettings zoals de achtergrond kunnen de settings opgeslagen worden in een persoonlijke map die gesynchroniseerd wordt met de inlog servers (roaming profile), maar voor applicaties is dit al een stuk lastiger, zeker als de verschillen per gebruiker groot zijn. Een veel gebruikte oplossing is dat op elk image alle gewenste applicaties aanwezig zijn en dat gebruikersrechten bepalen welke applicaties wel en niet zichtbaar zijn en opgestart kunnen worden.
\item Een non-persistent VDI zorgt ervoor dat bij elke nieuwe connectie de desktop eruit ziet zoals dat ingesteld is door de beheerder. Wijzigingen door gebruikers worden dus niet opgeslagen. Deze oplossing is goedkoper en eenvoudiger dan de non-persistent VDI oplossing.
\item er zijn ook tussen oplossingen waarbij gedeeltes worden opgeslagen in tussen oplossingen. Dit cre\"eert vaak snel complexiteit.
\end{itemize}
