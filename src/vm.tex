Machine virtualisatie zorgt ervoor dat er effici\"enter gebruik wordt gemaakt van computing resources. Moderne computers zijn zo krachtig dat ze met \'e\'en besturingssystemen en een paar applicaties of services vaak meer niets staan te doen dan dat ze werken. Door op deze hardware meerdere besturingssystemen te draaien kunnen de systemen effici\"enter ingezet worden. De grote datacenters hebben dan minder hardware nodig wat bezuinigt op stroomgebruik, warmte ontwikkeling beperkt en kastruimte bespaart.

Om meerdere besturingssystemen op \'e\'en machine te draaien is er iets nodig dat de bestaande hardware ophakt in stukjes zodat deze gezien wordt door de verschillende OSen als 'eigen' hardware. De techniek die daarvoor gebruikt wordt heet een hypervisor.
