VMWare kwam in 1999 met het product VMWare Player op de markt, vanaf 2015 heet het VMWare Workstation Player. VMWare Workstation Player draait op Windows en Linux en kan x86 systemen virtualiseren. VMWare Workstation Player is dan ook een Type 2 hypervisor, want het draait op een bestaand operating system. Er is ook een VMWare virtualisatie product op de markt voor Mac OS X systemen genaamd VMWare Fusion (beschikbaar vanaf 2006).

In 2001 kwam VMWare ESX op de markt, wat tegenwoordig VMWare ESXi heet, en dat is een type 1 hypervisor met zijn eigen besturingssysteem.

Het standaard formaat waarin VMWare zijn disk images schrijft heet VMDK - Virtual Machine DisK. Since versie 5.0 (2011) is VMDK een open format en wordt het ook door vele andere hypervisors ondersteund.
