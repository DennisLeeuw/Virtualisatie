Virtuele machines gebruiken virtuele netwerkkaarten en deze netwerkkaarten zijn gekoppeld aan virtuele netwerkswitches op de host machine. Die virtuele switches of vSwitches, zoals VMWare ze noemt, zijn niet zichtbaar omdat ze uit software zijn gemaakt, maar ze hebben wel effect op het functioneren van de machine. Zo kan een virtuele netwerkkaart direct gekoppeld zijn aan de netwerkkaart van de host en kan de virtuele machine direct met het lokale netwerk van de host praten. Het kan ook zo zijn dat de virtuele netwerkkaart in een virtuele switch zit waar alle andere virtuele machines ook in zitten en een virtuele adapter van het host systeem. Op die manier kan de host met de virtuele machines praten en kunnen de virtuele machines ook onderling met elkaar praten, maar via dit virtuele netwerk kan niemand met de buitenwereld praten.

Er zijn verschillende soorten netwerken die via hypervisors kunnen worden opgebouwd. In de volgende paragrafen zullen we de belangrijkste type netwerk configuraties doornemen.
