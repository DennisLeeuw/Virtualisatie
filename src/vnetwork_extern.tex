Een netwerk configuratie waarbij de VM alleen naar buiten kan wordt in Hyper-V een External netwerk genoemd en in VirtualBox een NAT netwerk. Dit is bij bijna alle hypervisors het standaard netwerk. De VM is ge\"isoleerd van de host en kan alleen via de netwerkkaart van de host naar het aangesloten netwerk. Een directe verbinding tussen host en VM is niet mogelijk.

Omdat de VM zijn eigen netwerk adressen heeft, al dan niet uitgedeeld door een DHCP-server op de host moeten de interne adressen via NAT (Network Address Translation) vertaald worden naar het externe IP adres van de netwerkkaart van de host om geen conflicten op te leveren op het aangesloten netwerk en om ervoor te zorgen dat antwoorden op connecties naar buiten weer terug worden gerouteerd naar de virtuele machine.
