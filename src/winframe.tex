In 1995 bracht Citrix Systems een product met de name WinFrame op de markt. WinFrame was Windows NT 3.51 als multi-user systeem. Meerdere gebruikers konden tegelijkertijd ingelogd zijn op het systeem en daar applicaties opstarten. De applicaties "draaiden" over het netwerk. De applicaties zijn dus niet lokaal ge\"installeerd, maar alleen ge\"installeerd op de server. Op de client is alleen de ICA client software ge\"installeerd, dit kan op een Windows systeem zijn, maar ook bijvoorbeeld een Mac OS X systeem of Linux. Op deze manier kunnen Windows applicaties gedeeld worden.

Om WinFrame te ontwikkelen heeft Citrix een licentie genomen op de Windows broncode voor NT 3.51 en heeft aan deze broncode de aanpassingen gedaan om van Windows NT een multi-user OS te maken. De verbinding tussen WinFrame en de desktop client wordt gemaakt via het ICA protocol. WinFrame heeft in de loop van de tijd wat naamsveranderingen ondergaan. Het heeft onderandere MetaFrame, Presentation Server en XenApp geheten. Het is tegenwoordig op de markt als Citrix Virtual Apps.

De client applicatie heette vroeger Citrix Receiver en heet nu Citrix Workspace App, voor Linux wordt er vaak gesproken over de ICAClient.

Microsoft heeft de code van Citrix WinFrame gelicenseerd van Citrix en heeft daarmee Microsoft Terminal Server op de markt gebracht in Windows NT 4.0. Vanaf Windows 2008 R2 (2009) heet het product Remote Desktop Services.
