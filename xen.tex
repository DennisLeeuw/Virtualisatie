Op de University of Cambridge begon men met een onderzoeksproject naar een type 1 hypervisor. Het project heette Xen en in 2003 kwam de eerste publieke versie op de markt. Een bedrijf werd opgericht XenSource Inc. die de open source versie beheerde en tevens een commercieel product op de markt bracht. Dit bedrijf werd in 2007 gekocht door Citrix. De open source versie werd ondergebracht bij xen.org. In 2013 kwam de open source Xen ontwikkeling onder de paraplu van de Linux Foundation en ging verder als het Xen Project (xenproject.org).

Citrix ontwikkelde, naast Citrix Xen Server, onder de Xen naam nog enkele andere productie die niets met de Xen hypervisor te maken hebben zoals XenApp en XenDesktop.

Ook andere leveranciers leveren de Xen Hypervisor als een commerieel product, bekende producten zijn:
\begin{itemize}
\item Huawei FusionSphere
\item Oracle VM Server for x86
\item Thinsy Corporation
\item Crucible (hypervisor) by Star Lab Corp
\end{itemize} 

